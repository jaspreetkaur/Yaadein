\documentclass{article}
\begin{document}
\section*{Documentation for making of souvenir}
In this documentation I tried to explain the procedure for making of souvenir. This procedure is after filling the data in the Souvenir(Yaadein)

\subsection*{Software requirements}
\begin{itemize}
\item Latex
\begin{itemize}
\item Installation: \$ sudo apt-get install texlive-full
\end{itemize}
\item sam2p : Image format converter
\begin{itemize}
\item Installation: \$ sudo apt-get install sam2p
\end{itemize}
\end{itemize}

\subsection*{Need of files and folders}
Following files and folders are required for making of souvenir.
\begin{enumerate}
\item Folders
\begin{itemize}
\item files: Previously it should be empty. After retreiving the images, folder contains images
\item headfoot: contains footer.eps and footer.eps is the image for footer and this can be designed using gimp or any other software
\end{itemize}
\item Files
\begin{itemize}
\item php files 

These are described below
\item final.tex

In this file, filebranch.tex files are included(like filecivil.tex, filecomputer.tex etc) and this is the file from which final.pdf will be produced.
\item Scripts

batchrename.php, cropper, eps, souvenir.run, souvenirnophp.run. All these are described below
\end{itemize}
\item sty files

lastpage.sty, wallpaper.sty, shapepar.sty. These are latex style files.
\item header eps files
sgeachcehead.eps, sgeachcsehead.eps, sgeachecehead.eps etc. These are header files which appear at the top of final.pdf file. These can be made with gimp or any other software
 \end{enumerate}

\subsection*{Retrieve the data from database}
After filling data in the souvenir, retrieve the data from the database. Following is the procedure for retrieving data:
\begin{itemize}
\item Require php files 
\begin{enumerate}
\item db.php
This file is for connectivity of database. You have to replace dbuser, dbpass, dbname created during installation of souvenir.
\item raiSED.php
This file is for defining the latex terms like \#, \%
\item querydav.php
This file is for selecting the data which is to be retreived
\item branch.php(branch can be civil, computer, it, electrical etc {like civil.php})
In these php files, db.php, raiSED.php and querydav.php are included. This file also explains the format of output.
\end{enumerate}
\end{itemize}

\subsection*{Get the images}
Images stored in the files folder (/usr/local/lib/python2.6/dist-packages/django/contrib/admin/media/
files) are retreived and convert these in the format `S2011classrollno.'. Procedure for this is:
\begin{enumerate}
\item Require batchrename.php
In this file, it is explained that in which format image is received. Before running this file you have to replace with own creates dbuser, dbpass and dbname
\begin{itemize}
\item Commands for to run batchrename.php file:

\$ php batchrename.php $>$retriever 

\$ chmod u+x reteriever

\$ ./retriever
\end{itemize}
\item Crop the images
For this run script `cropper' which is given in the package
\begin{itemize}
\item Commands to run cropper script:

\$ chmod u+x cropper

\$ ./cropper
\end{itemize}
\item Convert png format to eps of images
For this run script named `eps' which is given in package.
\begin{itemize}
\item Commands to run eps file:

\$ chmod u+x eps

\$ ./eps
\end{itemize}
\end{enumerate}

\subsection*{Compile php files for data retrieval}
Firstly run file souvenir.run(given in this package) which creates the filebranch.tex file(branch can be civil, computer, etc like {filecivil.tex}). Secondly run file souvenirnophp.run(also given in this package) which creates the final.pdf file.
\begin{itemize}
\item Commands to run the souvenir.run 

\$ chmod u+x souvenir.run

\$ ./souvenir.run

\item Commands to run souvenirnophp.run

\$ chmod u+x souvenirnophp.run

\$ ./souvenirnophp.run
\end{itemize}
final.pdf is the final output file.

\subsection*{Script}
Script for making of souvenir is also given in this package named `script.sh'. Before ruuning the script, following files and folders are required:
\begin{itemize}
\item empty folder named files
\item headfoot folder
\item branch.php files (civil.php, computer.php....etc)
\item db.php, raiSED.php, querydav.php
\item Scripts: batchrename.php, cropper, eps, souvenir.run, souvernirnophp.run
\item style files: wallpaper.sty, lastpage.sty, shapepar.sty
\item final.tex file
\end{itemize}
\begin{itemize}
\item Command to run script:

\$ sh script.sh
\end{itemize}

\end{document} 



